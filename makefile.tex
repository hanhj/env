%File Name:makefile.tex
%Created Time: 2019-04-27 11:44:42 week:6
%Author: hanhj
%Mail: hanhj@zx-jy.com 
\documentclass{article}
\usepackage{CJK}
\usepackage{graphicx}
\usepackage{indentfirst}%产生首行缩进
\usepackage{listings}
%\setlength{\parindent}{2em}
%\setlength{\parskip}{10pt}
\pagestyle{headings}
\lstset{language=C}
\begin{document}
\begin{CJK}{UTF8}{gbsn}
\title{Makefile文件使用说明}
\author{hanhj}
\maketitle
	\section*{前言}
	Makefile是一种脚本文件,被make命令所使用,用来自动执行一些命令,从而简化人们的工作。最常见的用途是用来编译程序,比如编译c或cpp文件。当然也可以用来做其他一些自动化工作,比如编译tex文件,生成pdf文档。编译doxygen文件,生成帮助文档等等。
	\par
	所以可以理解makefile文件就是一个用来实现自动化工作的脚本文件,与其他比如sh脚本文件一样的,只不过sh文件被bash所解释执行,而makefile文件被make所解释执行,二者有相同的地方,也有一些不同的地方,比如关于变量定义,使用都很相似,而在流程控制方面有些不同。此外,makefile也有一些自己的处理函数。
	\section{基本命令}
	\subsection {第一个makefile}
	\par
		假如我们有test.c文件,想把它编译成test程序。在命令行中,我们输入如下命令:gcc test.c,就可以了。但是,如果每次修改完test.c我们都要敲如上的字符,觉得太麻烦了,这时makefile就可以登场了,敲入如下代码,然后保存到Makefile文件中,这时,我们每次修改完test.c文件,直接敲make就可以得到test程序了。
	\begin{lstlisting}
	test:test.o
		gcc test.c -o test
	\end{lstlisting}
	注意第二行,前面需要用Tab键缩进。
	\par
		解释一下,第一行test:test.o,前面的test表示要生成的目标,后面的test.o表示为了生成这个目标所要依赖的文件,就是说为了生成test这个程序,需要test.o这个文件,这句话表示的就是这种依赖关系。  
	\par
	第二行gcc test.c -o test表示的是为了生成这个目标所要执行的动作,其实就是我们在命令行中所敲入的命令。不过需要注意的是在makefile文件中,描述这种动作的语句需要在前面用Tab键缩进。
	\par 
	Makefile的基本语法就是这样。
	\begin{lstlisting}
	target:depend file
		command
	\end{lstlisting}


\end{CJK}
\end{document}
