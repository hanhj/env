\documentclass[fleqn]{article}
\usepackage{xeCJK}
\usepackage{graphicx}
\usepackage{indentfirst}%产生首行缩进
\usepackage{listings}
\usepackage{makeidx}
\usepackage{mylatex}
\usepackage[breaklinks,colorlinks,linkcolor=black,citecolor=black,urlcolor=black]{hyperref}
\setCJKmainfont{AR PL UMing TW MBE}
\thispagestyle{headings}
\lstset{language=C}
\makeindex
\begin{document}
%\begin{CJK}{UTF8}{gbsn}
\title{\LaTeX 使用简要说明}
\author{hanhj}
\maketitle
\newpage
\tableofcontents
\newpage
\section{安装latex软件}
	\par 
	为了使用latex需要安装如下软件:
	\begin{verbatim}
	texlive									:基础
	texlive-xetex							:编译器
	tex-cjk-chinese,texlive-lang-chinese	:中文支持
	texmaker:latex							:latex编辑器,可选
	帮助:
		texdoc 包名,可以查看本地包的帮助信息。
		本地的包放在两处,一处是发行的包放在 /usr/share/texlive/texmf-dist/tex/latex/。
		一处是插件,放在 /usr/share/texmf/tex/latex/下面。
	\end{verbatim}
\section {第一个latex文档}
	\begin{verbatim}
		\documentclass{article}
		\begin{document}
			Hello world
		\end{document}
	\end{verbatim}
	将以上文档保存为test.tex \\
	编译:pdflatex test.tex	默认输出pdf\\
	在linux下查看pdf文档有evince,okular,xpdf等\\ 
\section{中文支持}
	\begin{itemize}
	\item 使用CJK来支持中文:CJK是chinese,Japan,Korea中日韩三国。CJK是德国人写的一种中文挂件。
		\begin{verbatim}
			\documentclass{article}
			\usepackage{CJK}			%使用CJK
			\begin{CJK}{UTF8}{gbsn}		%指定编码UTF8,字体gbsn
			\begin{document}
				你好,世界
			\end{CJK}
			\end{document}
		\end{verbatim}
	编译:pdflatex test.tex 
	\par
	关于编译:latex文档的编译工具有latex,xelatex,pdflatex.\\
	传统的方法:先用latex生成dvi文件,然后用dvips转成ps文件,再用ps2pdf转成pdf文件。\\
	latex xx.tex\\
	dvips xx.dvi \\
	ps2pdf xx.ps
	\par 
	第二步也可省略,用dvipdf直接将dvi文件转换成pdf,dvipdf xx.dvi。
	\par
	更直接的方法是用pdflatex,直接将tex文件转换成pdf。pdflatex xx.tex
	\par
	另外一种直接的方法是用xelatex,也可直接将tex文件转换成pdf。但是这种方法需要在文档中使用xetex或ctex来支持中文(参见下文)。xelatex xx.tex 
	\item 使用xeTex来支持中文 
		\begin{verbatim}
			\documentclass{article}
			\usepackage{xeCJK}
			\setCJKmainfont{AR PL UMing CN} %设置中文字体
			\begin{document}
				你好,世界
			\end{document}
		\end{verbatim}
	编译:xelatex test.tex 	
	\item 使用ctex来支持中文。ctex是中国人写的从底层支持中文的一种方法。	
		\begin{verbatim}
		\documentclass{ctexart}
		\usepackage{ctex}
		\begin{document}
		...
		\end{document}
		\end{verbatim}
	\par 
	编译:xelatex test.tex。
	\par
	\item	中文字体
		\begin{itemize} 
		\item 当使用CJK时\\可选用的字体在:/usr/share/texmf/tex/latex/CJK/UTF8/ 目录下,有bkai楷体,gbsn宋体,gkai楷体,bsmi,goth,min。
		\begin{verbatim}
			\begin{CJK}{UTF8}{gbsn}		%指定编码UTF8,字体gbsn。
		\end{verbatim}
		在这里可以选择不同字体,如gkai,bkai等。
		\item 当使用ctex时\\ctex内置支持几种字体宋体,黑体,仿宋,楷书,在 /usr/share/texlive/texmf-dist/tex/latex/ctex/fontset/ctex-fontset-fandol.def中定义了一些预定义:
			\begin{verbatim} 
				\songti \heiti \fangsong \kaishu 
			\end{verbatim} 
			可以在document之前使用这些预定义。例如:
			\begin{verbatim}
				\documentclass{ctexart}
				\usepackage{ctex}
				\heiti
				\begin{document}
				...
				\end{document}
				注意:需要用xelatex来编译。
			\end{verbatim}
		\item 另外可以使用ubuntu中的系统字体。\\
		首先:用fc-list :lang=zh 查找当前系统中的中文字体。然后使用
			\begin{verbatim}
				setCJKmainfont{字体}来设置。
			\end{verbatim}
			比如fc-list命令输出有下面一行:	/usr/share/fonts/truetype/arphic/uming.ttc: AR PL UMing TW MBE:style=Light 。然后在tex文件中加入:
			\begin{verbatim}
				\documentclass{article}
				\usepackage{xeCJK}
				setCJKmainfont{AR PL uming TW MBE} %就可以使用这种字体了。
				\begin{document}
				...
				\end{document}
				注意:需要用xelatex编译。
			\end{verbatim}
		\end{itemize}
	\end{itemize}
\section{文档结构}
	你现在已经按照上面的说明,建立第一个latex文档,并支持中文了。现在解释一下第一个文档:
	\begin{verbatim}
		\documentclass{article}
		\begin{document}
			Hello world
		\end{document}
	\end{verbatim}
	Latex文档由控制命令和正文组成。其中控制命令由$\backslash$后面跟一个命令组成。比如:
	\begin{verbatim}
	\documentclass{article}
	\end{verbatim}
	命令只由字母组成,latex以命令之后的空格,数字或非字母的字符作为该命令的结束。命令可以带参数,用\{...\}括起来,也可以带选项[...],比如:
	\begin{verbatim}
		\documentclass[UTF8]{article} %表示文档采用utf8编码
	\end{verbatim}
	\subsection{文档结构}
	\[	
	\textrm{引言部分}\left\{ 
			\begin{array}{ll}
				\backslash documentclass\{article\} & \textrm{这个是定义文档格式}\\
				  \backslash usepackage\{graphicx\} & \textrm{这个是引用宏包}\\
																		 ...& \textrm{其他一些宏包} \\
				... 
			\end{array}
			\right.
	\]
	\[
	\textrm{正文部分}\left\{
			\begin{array}{ll}
				\backslash begin\{document\} & \textrm{表示正文开始}\\
				\backslash title\{...\} & \textrm{标题}\\
				\backslash maketitle & \textrm{生成标题}\\
				...\\
				\backslash end\{document\} & \textrm{表示正文结束}
			\end{array}
		\right.
	\]
	文档类documentclass的参数包括:article(论文,期刊),proc(会议文集),report,book等。\\
	其配置项包括:fleqn(行间公式左对齐,默认中间对齐),10pt,11pt(文中字体大小,默认10pt)等等。\\
	宏包是为了增强latex的功能所发布的命令包。比如通过graphicx包可以插入图片。\\
	命令是对全文的一些处理:比如setCJKmainfont设置字体,maketitle设置文档首页内容(包括标题,作者,日期)。

	\subsection{保留字符}
		在latex中有一些字符是作为保留用的特殊字符,通常在正文中不会被打印出来:
		\begin{table}[!htp]
		\[	
			\begin{array}{l|l}
				\hline
				\#	& \textrm{自定义命令中的参数}\\ 
				\hline
				\&	&		\textrm{代表表格中的连接符}\\ 
				\hline
				\%	& \textrm{代表注释}\\
				\hline
				\$	&\textrm{代表数学符号}\\
				\hline
				\{ \} &\textrm{ 代表参数}\\
				\hline
				\wedge	&\textrm{代表上标}\\
				\hline
				\_	&\textrm{代表下标}\\
				\hline
				\backslash &\textrm{代表命令}	\\
				\hline
			\end{array}
		\]
			\caption{保留字符}
		\end{table}
		\par
		如果在正文中想打印这些字符,在前面加上$\backslash$,特别的对于$\backslash$要用$\backslash$backslash来表示,$\wedge$要用$\backslash$wedge来表示。
\section{基本排版}
	\subsection{标题}\index{title}
		正文的标题可以用下面的文字生成:
		\begin{verbatim}
			\begin{document}
				\title{文章的标题}
				\author{作者}
				\today
				\maketitle
				...
			\end{document}
		\end{verbatim}
		这样就可以生成标题了。\ldots 是正文其他部分。
	\subsection{节}\index{section}
		节是内容的比较大的集中部分
		\begin{verbatim}
			\section{节的标题}
		\end{verbatim}
		节会产生编号(如果在section后跟*则不会产生编号)。在节中还可以分成字节,子子节,分别用$\backslash$subsection,和$\backslash$subsubsection来产生。
	\subsection{段}\index{par}	
		段是一段内容的集合,是比节小一点的内容的集合。
		\begin{verbatim}
			\par 
		\end{verbatim}
		在两段之间空两行,也可以达到分段的目的。除非强制分行,否则在一段内只有一行,不会另起一行。\\
		如果想强制分行,可以在行尾加上$\backslash$$\backslash$。或者$\backslash$newline.\\
		段的开始,根据设定可以产生缩进。可以手动设置:
		\begin{verbatim}
			\setlength{\parindent}{20em}
		\end{verbatim}
		或者用引言部分用包$\backslash$usepackage\{indentfirst\}。 \\
	\subsection{其他一些分段命令}\index{paragraph}\index{part}\index{newpage}\index{linebreak}\index{pagebreak}
		paragraph:用来产生一段,与section类似,但是不会产生编号。\\
		part:用来分部。比section更大的分段,它不会影响部与部之间的节编号。\\
		newpage:另起一页。\\
		linebreak[n],nolinebreak[n]:新增n行\\
		pagebreak[n],nopagebreak[n]:新增n页
	\subsection{强调}\index{emph}
		为了突出某些内容,可能需要对文字作出一些强调。强调包括下划线,斜体,黑体。
		\begin{verbatim}
			\underline{...} 下划线
			\emph{...}		斜体 
			\textit{...}	这也是斜体
			\textbf{...}	黑体
		\end{verbatim}
		\underline{下划线},\emph{斜体,hello},\textbf{黑体}

	\subsection{交叉引用}\index{label}
		在文档内可能需要对某些内容做标记,当别处引用此内容时,可以快速能够跳到该处。
		\begin{verbatim}
			\label{mark}	在需要做标记的地方
			\ref{mark}		在引用的地方,显示标记号
			\pageref{mark}	在引用的地方,显示标记的页号
		\end{verbatim}
	\subsection{脚注}\index{footnote}
		\begin{verbatim}
			\footnote{...}	在需要做脚注的地方
		\end{verbatim}
		\footnote{this is a footnote}脚注
	\subsection{摘要}\index{abstract}
		\begin{verbatim}
			论文中常在开头做摘要
			\abstract{...}
		\end{verbatim}
		

	\subsection{环境}\index{begin}\index{end}	
		前面描述的命令都是单条命令,在latex中还有一种叫做环境的命令,它用begin开头,end结尾.形如:
		\begin{verbatim}
			\begin{command}
				...
			\end{command}
		\end{verbatim}
		意思是这条命令将影响到所包括的内容。document就是一条环境命令。除此以外还有如下的一些常用的环境命令:
		\begin{table}[!hbp]
		\[
			\begin{array}{l|l}
				\hline
					flushleft,flushright &\textrm{左对齐,右对齐}\\
				\hline
					itemize &\textrm{无序列表}\\
				\hline
				    enumerate &\textrm{有序列表}\\
				\hline
					tabular &\textrm{表格}\\
				\hline
					array &\textrm{数学公式中的多行公式}	\\
				\hline
					verbatim &\textrm{代码}\\	
				\hline
			\end{array}
		\]
			\caption{常用环境命令}
		\end{table}	
		

		\underline{itemize用法:}\index{itemize}
		\begin{verbatim}
			\begin{itemize}
					\item xxx
					\item xxx
					...
			\end{itemize}
		\end{verbatim}
		\underline{enumerate用法:}\index{enumerate}
		\begin{verbatim}
			\begin{enumerate}
					\item xxx
					...
			\end{enumerate}
		\end{verbatim}
		\underline{verbatim用法:}\index{verbatim}
		\begin{verbatim}
			\begin{verbatim}
				随便写,就像code一样
			\end{verbatim }
		\end{verbatim}
		\underline{tabular用法:}\index{tabular}\index{hline}\index{cline}
		\begin{verbatim}
		\begin{tabular}[文字位置]{列对齐}
			row1_column1 & row1_column2 ...
			...
			\end{tabular}
		\end{verbatim}
		下面是一段示例:\\
		\begin{tabular}[t]{|l|l|}
			\hline
			A & B \\
			\hline
		   a1 & b1 \\
			\hline
		   a2& b2\\
			\hline
		\end{tabular}
		\begin{verbatim}
			\begin{tabular}[t]{|l|l|}
			\hline
			A & B \\
			\hline
			a1 & b1 \\
			\hline
			a2& b2\\
			\hline
			\end{tabular}
		\end{verbatim}
		文字位置用t表示对上对齐,b表示对下对齐,c表示居中。\\
		列对齐用l表示左对齐,r表示右对齐,c表示居中。中间用$\mid$表示画竖线。\\
		hline表示画横线,cline\{n,n\}表示在第n列中画竖线。\\
		\underline{array用法:}\\
		下面是一段示例:
		\[
			\left\{ 
				\begin{array}{ll}
					I_x<dz\\
					U_x<dz 
				\end{array}
			\right.
		\]
	\begin{verbatim}
	\[
		\left\{ 
			\begin{array}{ll}
				I_x<dz\\
				U_x<dz 
			\end{array}
		\right.
	\]
	\end{verbatim}
	\subsection{数学公式}\index{math}
		数学公式在latex中有两种打印方式:行内或行间。行内就是在一行之内,用\$...\$来表示。行间就是位于两行之间,用
		\begin{verbatim}
			\[
			...
			\]
		\end{verbatim}
		或者用
		\begin{verbatim}
			\begin{math}
				...
			\end{math}
		\end{verbatim}
		或者用 
		\begin{verbatim}
			\begin{equation}
				...
			\end{equation}
		\end{verbatim}
		比如:$a^2+b^2=c$这是行内公式。
		\[
			x^2+y^2=z 
		\]
		这是行间公式。\\
		math与equation的区别在于equation会自动给公式加上编号(默认在右侧),而math不会。\\
		数学公式的写法基本与自然写法一样,只不过要记住公式中各个元素的表达方式。比如$\alpha$用$\backslash$alpha来表示,$\sum$用$\backslash$sum来表示。具体元素的表达方法参见\cite{p1},多用几次就熟悉了。这里提几个修饰符:\\
		\begin{table}[!htp]
		\begin{tabular}[t]{l|l}
			\hline
			$\wedge$ &上标,可以用来表示次方,积分或求和公式的上限。\\
			\hline
			$\_$  &下标,可以用来表示公式中的下标,积分或求和公式中的下限。\\
			\hline
			$\overline{m+n}$ & 上划线,$\backslash$overline \\
			\hline
			$\underline{m+n}$ & 下划线,$\backslash$underline \\
			\hline
			$\overbrace{a+b+c}$ & 上括号,$\backslash$overbrace\\ 
			\hline
			$\underbrace{a+b+c}$ & 下括号,$\backslash$underbrace\\ 
			\hline
			$\frac{a}{b}$ & 分号,$\backslash$frac\{a\}\{b\}\\
			\hline
			空格 &$\backslash$quad or $\backslash$qquad\\
			\hline
			公式中的字串 & $\backslash$mathrm\{...\} \\
			\hline
		\end{tabular}
			\caption{数学公式}
		\end{table}	
	\subsection{表和图}\index{table}\index{figure}
		文档中常有表和图,需要给它们编号,table和figure就是为表和图加上编号的命令。table,figure在latex中称为浮动体。
		\begin{verbatim}
			\begin{figure}[option]
				插入图片
			\end{figure}

			\begin{table}[option]
				插入表格
			\end{table}
		\end{verbatim}
		使用默认的figure往往不能将图片放到期望的位置。这时可以用选项[!htp]。
		\begin{table}[!htp]
			\begin{tabular}{l|l}
				\hline
				h & here 明确的位置\\
				\hline
				t & top	页面顶部\\
				\hline
				b & bottom	页面底部\\
				\hline
				p & page 在一个只有浮动体的页面\\
				\hline
				! & 取消大多数浮动体的默认设置\\
				\hline
			\end{tabular}
			\caption{浮动体位置}
		\end{table}\\
		为了插入图片,需要使用graphicx包。
		\begin{verbatim}
			usepackage{graphicx}
			\begin{document}
			...
				\includegraphics[option]{filename} %这里插入图片

			\end{document}
		\end{verbatim}
		filename可以不带后缀名,让latex自己在当前目录中找图片。如果用latex编译,支持的文件格式为eps。如果用pdflatex编译,支持的文件格式包括png,pdf,jpg,mps。不幸的是pdflatex并不支持eps,所以如果用pdflatex来编译,需要将eps文件转换成其所支持的文件。\\
		includegraphics的选项包括:
		\begin{table}[!hp]
			\begin{tabular}{l|l}
				\hline
				scale &	缩放\\
				\hline
				width & 宽度\\
				\hline
				height & 高度\\
				\hline
				angle & 角度\\
				\hline
			\end{tabular}
			\caption{graphicx的选项}
		\end{table}
		\par
		高度和宽度的单位可以是:mm,cm,in(1英寸=25.4mm),pt(point $1point\approx\frac{1}{3}mm $),em(当前字母M的宽度),ex(当前字母x的高度)。\label{unit}
		\par
		表和图的索引可以用:
		\begin{verbatim}
			\listoftables
			\listoffigures
		\end{verbatim}
		来生成,一般放在文档最后。
		\subsection{索引}\index{index}
		在文档中对一些关键词进行索引,将该索引表放在文档的最后,可以方便的在文档中查找关键词是非常有用的。
		\par
		为了生成索引,需要在引言部分引入makeidx包,并且调用makeindex命令。在正文中需要产生索引的地方调用index命令,在需要打印索引表的地方调用printindex命令。
		\begin{verbatim}
			\usepackage{makeindex}	%引入makeidx包
			...
			\makeindex				%调用makeindex命令,生成index
			\begin{document}
			...
				\index{key}		%产生索引
			...

				\printindex		%打印索引表
			\end{document}
		\end{verbatim}
		在编译的时候,需要两次编译。即latex xx.tex,makeindex xx.idx,latex xx.tex。第一次编译生成索引文件idx,用makeindex程序将idx文件转换成ind文件,第二次编译的时候就会将ind文件包括进来,在printindex处打印。
		\subsection{引用}\index{thebibliography}
		论文中经常有引用的参考文献,用如下方法实现:\\
		定义参考文献:
		\begin{verbatim}
			\begin{thebibliography}
					\bibitem{标号} 文献描述
					...
		\end{verbatim}
		然后在引用的地方:
		\begin{verbatim}
			\cite{标号}
		\end{verbatim}
		\subsection{在pdf文档中加入左侧书签}
		需要引入hyperref宏包。由于hyperref使用了扩展CJK,所以需要用xelatex来支持中文。
		\begin{verbatim}
			\usepackage{xeCJK}
			\setCJKmainfont{AR PL UMing CN} %宋体,可以选用其他字体
			\usepackage{hyperref}
			...
		\end{verbatim}
\section{扩展latex}
	以上是latex的基本用法,为了提高效率,增加功能有必要对基本latex进行扩展。	\subsection{用户自定义命令}
		为了提高效率,可以将常用的命令定义成用户自定义命令,从而可以减少敲键盘的次数。用户自定义命令用newcommand,renewcommand来定义。前者适合原系统没有定义的命令,后者适合系统已经定义的命令,用来重新定义。(类似于vim中的map)命令。命令定义需要放在引言处。
		\begin{verbatim}
			\newcommand{\command}[n]{define}
			\renewcommand{\command}[n]{define}
		\end{verbatim}
		command是新定义的命令名,define是需要执行的动作,n是参数个数,参数个数从1到9,如果没有定义,默认是0。参数从命令后开始1,2...。在定义中引用参数使用\#1,\#2...。下面是一个例子:假设将$\backslash$la定义成\LaTeX
		\begin{verbatim}
			\newcommand{\la}[1]{\LaTaX{#1}}
		\end{verbatim}
		这是使用$\backslash$laAAA的效果\la{AAA} .
	\subsection{用户自定义环境命令}
		\begin{verbatim}
			\newenvironment{command}[n]{action on begin}{action after end}
		\end{verbatim}
		下面的例子假设在输入字串的两边加上$[]$
		\begin{verbatim}
			\newenvironment{tl}
				{
					\textbf{[}%定义begin动作
				}
				{
					\textbf{]}%定义end动作
				}
		\end{verbatim}
		这里使用了tl环境命令
		\begin{verbatim}\begin{tl}Hello\end{tl}\end{verbatim}的效果:\begin{tl}Hello\end{tl}
	\subsection{定义用户自己的包}
		上述命令放在导言区,每次需要包括比较麻烦,可以类似c的文件,将上述内容放在一个sty文件中,然后使用usepackage命令包括进来就比较方便了。

		sty文件内容与上述内容类似,只不过需要在第一句加上ProvidesPackage\{包名\}。下面是一个sty文件的例子:
		\begin{verbatim}
			\ProvidesPackage{mylatex}
			\newsavebox{\newname}
			\newcommand{\la}[1]{\LaTeX\textbf{#1}}
			\newenvironment{tl}
				{
					\textbf{[}%
				}
				{
					\textbf{]}%
				}
		\end{verbatim}
		该文件放在编译文件的同一个目录,或者放在/.texmf/tex/latex目录下面。同时设置环境变量TEXMFHOME:export TEXMFHOME=/.texmf/tex/latex。上面的目录是用户私有的,也可以放在系统的包目录:\mbox{/usr/share/texmf/tex/latex中}。
	\subsection{其他包}
		在网络上www.ctan.org网站上可以搜索到其他有用的包。\\
		从上面下载包到本地,一般是压缩文件。解压文件到 /usr/share/texmf/tex/latex/目录下。包里面有ins文件,是安装文件,用latex来编译,可以生成必要的sty等文件。如果没有ins文件,也会有dtx文件是发行文件,用latex编译该文件会生成sty文件,这个就是我们需要的包文件。\\
		执行mktexlsr,或texhash命令更新latex文件名数据库,就可以使用了。\\
		\textbf{生成帮助:}\\
			用latex来编译dtx文件,然后会生成dvi文件,再用dvips转成pdf文件,必要的话可能还需要makeindex来生成索引文件。以上是生成帮助的手动方法,有些包将以上过程封装在一个脚本中,也是一样的。


\newpage
\begin{thebibliography}{2}
		\bibitem{p1} latex官方文档
\end{thebibliography}
\newpage
\listoftables
\printindex
%\end{CJK}
\end{document}

