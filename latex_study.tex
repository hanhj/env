\documentclass[fleqn]{article}
\usepackage{CJK}
\usepackage{graphicx}
%\usepackage{indentfirst}%产生首行缩进
\usepackage{listings}
\thispagestyle{headings}
\lstset{language=C}
\begin{document}
\begin{CJK}{UTF8}{gbsn}
\title{\LaTeX 使用简要说明}
\author{hanhj}
\maketitle
\section{安装latex软件}
	\par 
	为了使用latex需要安装如下如软件:
	\begin{verbatim}
	texlive									:基础
	texlive-xetex							:编译器
	tex-cjk-chinese,texlive-lang-chinese	:中文支持
	texmaker:latex							:latex编辑器,可选
	\end{verbatim}
	
\section {第一个latex文档}
	\begin{verbatim}
		\documentclass{article}
		\begin{document}
			Hello world
		\end{document}
	\end{verbatim}
	将以上文档保存为test.tex \\
	编译:pdflatex test.tex	默认输出pdf\\
	在linux下查看pdf文档有evince,okular,xpdf等\\ 
\section{中文支持}
	\begin{itemize}
	\item 使用CJK来支持中文:CJK是chinese,Japan,Korea中日韩三国。CJK是德国人写的一种中文挂件。
		\begin{verbatim}
			\documentclass{article}
			\usepackage{CJK}			%使用CJK
			\begin{CJK}{UTF8}{gbsn}		%指定编码UTF8,字体gbsn
			\begin{document}
				你好,世界
			\end{CJK}
			\end{document}
		\end{verbatim}
	编译:pdflatex test.tex 
	\par
	关于编译:latex文档的编译工具有latex,xelatex,pdflatex.\\
	传统的方法:先用latex生成dvi文件,然后用dvips转成ps文件,再用ps2pdf转成pdf文件。\\
	latex xx.tex\\
	dvips xx.dvi \\
	ps2pdf xx.ps
	\par 
	第二步也可省略,用dvipdf直接将dvi文件转换成pdf,dvipdf xx.dvi。
	\par
	更直接的方法是用pdflatex,直接将tex文件转换成pdf。pdflatex xx.tex
	\par
	另外一种直接的方法是用xelatex,也可直接将tex文件转换成pdf。但是这种方法需要在文档中使用xetex或ctex来支持中文(参见下文)。xelatex xx.tex 
	\item 使用xeTex来支持中文 
		\begin{verbatim}
			\documentclass{article}
			\usepackage{xeCJK}
			\setCJKmainfont{AR PL UMing CN} %设置中文字体
			\begin{document}
				你好,世界
			\end{document}
		\end{verbatim}
	编译:xelatex test.tex 	
	\item 使用ctex来支持中文。ctex是中国人写的从底层支持中文的一种方法。	
		\begin{verbatim}
		\documentclass{ctexart}
		\usepackage{ctex}
		\begin{document}
		...
		\end{document}
		\end{verbatim}
	\par 
	编译:xelatex test.tex。
	\par
	\item	中文字体
		\begin{itemize} 
		\item 当使用CJK时\\可选用的字体在:/usr/share/texmf/tex/latex/CJK/UTF8/ 目录下,有bkai楷体,gbsn宋体,gkai楷体,bsmi,goth,min。
		\begin{verbatim}
			\begin{CJK}{UTF8}{gbsn}		%指定编码UTF8,字体gbsn。
		\end{verbatim}
		在这里可以选择不同字体,如gkai,bkai等。
		\item 当使用ctex时\\ctex内置支持几种字体宋体,黑体,仿宋,楷书,在 /usr/share/texlive/texmf-dist/tex/latex/ctex/fontset/ctex-fontset-fandol.def中定义了一些预定义:
			\begin{verbatim} 
				\songti \heiti \fangsong \kaishu 
			\end{verbatim} 
			可以在document之前使用这些预定义。例如:
			\begin{verbatim}
				\documentclass{ctexart}
				\usepackage{ctex}
				\heiti
				\begin{document}
				...
				\end{document}
				注意:需要用xelatex来编译。
			\end{verbatim}
		\item 另外可以使用ubuntu中的系统字体。\\
		首先:用fc-list :lang=zh 查找当前系统中的中文字体。然后使用
			\begin{verbatim}
				setCJKmainfont{字体}来设置。
			\end{verbatim}
			比如fc-list命令输出有下面一行:	/usr/share/fonts/truetype/arphic/uming.ttc: AR PL UMing TW MBE:style=Light 。然后在tex文件中加入:
			\begin{verbatim}
				\documentclass{article}
				\usepackage{xeCJK}
				setCJKmainfont{AR PL uming TW MBE} %就可以使用这种字体了。
				\begin{document}
				...
				\end{document}
				注意:需要用xelatex编译。
			\end{verbatim}
		\end{itemize}
	\end{itemize}
\section{基础}
	你现在已经按照上面的说明,建立第一个latex文档,并支持中文了。现在解释一下第一个文档:
	\begin{verbatim}
		\documentclass{article}
		\begin{document}
			Hello world
		\end{document}
	\end{verbatim}
	Latex文档由控制命令和正文组成。其中控制命令由$\backslash$后面跟一个命令组成。比如:
	\begin{verbatim}
	\documentclass{article}
	\end{verbatim}
	命令只由字母组成,latex以命令之后的空格,数字或非字母的字符作为该命令的结束。命令可以带参数,用\{...\}括起来,也可以带选项[...],比如:
	\begin{verbatim}
		\documentclass[UTF8]{article} %表示文档采用utf8编码
	\end{verbatim}
	\subsection{文档结构}
	\[	
	\textrm{引言部分}\left\{ 
			\begin{array}{ll}
				\backslash documentclass\{article\} & \textrm{这个是定义文档格式}\\
				  \backslash usepackage\{graphicx\} & \textrm{这个是引用宏包}\\
																		 ...& \textrm{其他一些宏包} \\
				  \backslash maketitle & \textrm{其他一些命令}\\
				... 
			\end{array}
			\right.
	\]
	\[
	\textrm{正文部分}\left\{
			\begin{array}{ll}
				\backslash begin\{document\} & \textrm{表示正文开始}\\
				...\\
				\backslash end\{document\} & \textrm{表示正文结束}
			\end{array}
		\right.
	\]
	文档类documentclass的参数包括:article(论文,期刊),proc(会议文集),report,book等。\\
	其配置项包括:fleqn(行间公式左对齐,默认中间对齐),10pt,11pt(文中字体大小,默认10pt)等等。
	宏包是为了增强latex的功能,所发布的命令包。比如通过graphicx包可以插入图片。\\
	命令是对全文的一些处理:比如setCJKmainfont设置字体,maketitle设置文档首页内容(包括标题,作者,日期)。

	\subsection{保留字符}
		在latex中有一些字符是作为保留用的特殊字符,通常在正文中不会被打印出来:
		\[	
			\begin{array}{l|l}
				\hline
				\#	&		\\ 
				\hline
				\&	&		\textrm{代表表格中的连接符}\\ 
				\hline
				\%	& \textrm{代表注释}\\
				\hline
				\$	&\textrm{代表数学符号}\\
				\hline
				\{ \} &\textrm{ 代表参数}\\
				\hline
				\wedge	&\textrm{代表上标}\\
				\hline
				\_	&\textrm{代表下标}\\
				\hline
				\backslash &\textrm{代表命令}	\\
				\hline
			\end{array}
		\]
		如果在正文中想打印这些字符,在前面加上$\backslash$,特别的对与$\backslash$要用$\backslash$backslash来表示,$\wedge$要用$\backslash$wedge来表示。
\section{排版}
	\subsection{标题}
		正文的标题可以用下面的文字生成:
		\begin{verbatim}
			\begin{document}
				\title{文章的标题}
				\author{作者}
				\today
				\maketitle
				...
			\end{document}
		\end{verbatim}
		这样就可以生成标题了。\ldots 是正文其他部分。
	\subsection{节}
		节是内容的比较大的集中部分
		\begin{verbatim}
			\section{节的标题}
		\end{verbatim}
		节会产生编号(如果在section后跟*则不会产生编号)。在节中还可以分成字节,子子节,分别用$\backslash$subsection,和$\backslash$subsubsection来产生。
	\subsection{段}	
		段是一段内容的集合,是比节小一点的内容的集合。用$\backslash$par来产生。除非强制分行,否则内容在一段内是连续的,不会另起一行。在两段之间空两行,也可以达到分段的目的。\\
		段的起始,根据设定可以产生缩进。可以手动设置:\begin{verbatim}
			\setlength{\parindent}{20em}
		\end{verbatim}或者用包$\backslash$usepackage\{indentfirst\}。 \\
		强制分行,可以在行尾加上$\backslash$$\backslash$。或者$\backslash$newline.
	\subsection{其他一些分段命令}
		paragraph:用来产生一段,与section类似,但是不会产生编号。\\
		part:用来分部。比section更大的分段,它不会影响部与部之间的节编号。\\
		newpage:另起一页。\\
		linebreak[n],nolinebreak[n]:新增n行\\
		pagebreak[n],nopagebreak[n]:新增n页
	\subsection{环境}	
		前面描述的命令都是单条命令,在latex中还有一种叫做环境的命令,它用begin开头,end结尾.形如:
		\begin{verbatim}
			\begin{command}
				...
			\end{command}
		\end{verbatim}
		意思是这条命令将影响到所包括的内容。document就是一种环境变量。除此以外还有如下的一些常用的环境命令:
		\[
			\begin{array}{l|l}
				\hline
					itemize &\textrm{无序列表}\\
				\hline
				    eumerate &\textrm{有序列表}\\
				\hline
					tabular &\textrm{表格}\\
				\hline
					array &\textrm{数学公式中的多行公式}	\\
				\hline
					verbatim &\textrm{代码}\\	
				\hline
			\end{array}
		\]
		

		\underline{itemize用法:}
		\begin{verbatim}
			\begin{itemize}
					\item xxx
					\item xxx
					...
			\end{itemize}
		\end{verbatim}
		\underline{enumerate用法:}
		\begin{verbatim}
			\begin{enumerate}
					\item xxx
					...
			\end{enumerate}
		\end{verbatim}
		\underline{verbatim用法:}
		\begin{verbatim}
			\begin{verbatim}
				随便写,就像code一样
			\end{verbatim }
		\end{verbatim}
		\underline{tabular用法:}
		\begin{verbatim}
		\begin{tabular}[文字位置]{列对齐}
			row1_column1 & row1_column2 ...
			...
			\end{tabular}
		\end{verbatim}
		下面是一段示例:\\
		\begin{tabular}[t]{|l|l|}
			\hline
			A & B \\
			\hline
		   a1 & b1 \\
			\hline
		   a2& b2\\
			\hline
		\end{tabular}
		\begin{verbatim}
			\begin{tabular}[t]{|l|l|}
			\hline
			A & B \\
			\hline
			a1 & b1 \\
			\hline
			a2& b2\\
			\hline
			\end{tabular}
		\end{verbatim}
		文字位置用t表示对上对齐,b表示对下对齐,c表示居中。\\
		列对齐用l表示左对齐,r表示右对齐,c表示居中。中间用$\mid$表示画竖线。\\
		hline表示画横线,cline\{n,n\}表示在第n列中画竖线。\\
		\underline{array用法:}\\
		下面是一段示例:
		\[
		\left\{ 
			\begin{array}{ll}
				I_x<dz\\
				U_x<dz 
			\end{array}
		\right.
	\]
	\begin{verbatim}
	\[
		\left\{ 
			\begin{array}{ll}
				I_x<dz\\
				U_x<dz 
			\end{array}
		\right.
	\]
	\end{verbatim}

	\subsection{引用}
		论文中经常有引用的参考文献,用如下方法实现:\\
		定义参考文献:
		\begin{verbatim}
			\begin{thebibliography}
					\bibitem{标号} 文献描述
					...
		\end{verbatim}
		然后在引用的地方:
		\begin{verbatim}
			\cite{标号}
		\end{verbatim}
		
		

\begin{thebibliography}{2}
		\bibitem{p1} latex官方文档
\end{thebibliography}		

\end{CJK}
\end{document}

